\documentclass[journal,12pt,twocolumn]{IEEEtran}
%
\usepackage{setspace}
\usepackage{gensymb}
%\doublespacing
\singlespacing

\usepackage{graphicx}
\usepackage[cmex10]{amsmath}
\usepackage{amsmath,amsthm}
\usepackage{mathrsfs}
\usepackage{txfonts}
\usepackage{stfloats}
\usepackage{bm}
\usepackage{cite}
\usepackage{cases}
\usepackage{subfig}

\usepackage{longtable}
\usepackage{multirow}
\usepackage{commath}
\usepackage{enumitem}
\usepackage{mathtools}
\usepackage{steinmetz}
\usepackage{tikz}
\usepackage{circuitikz}
\usepackage{verbatim}
\usepackage{tfrupee}
\usepackage[breaklinks=true]{hyperref}

\usepackage{tkz-euclide}

\usetikzlibrary{calc,math}
\usepackage{listings}
\usepackage{color}                                            
\usepackage{array}                                            
\usepackage{longtable}                                        
\usepackage{calc}                                             
\usepackage{multirow}                                         
\usepackage{hhline}                                           
\usepackage{ifthen}                                           
\usepackage{lscape}     
\usepackage{multicol}
\usepackage{chngcntr}

\DeclareMathOperator*{\Res}{Res}

\renewcommand\thesection{\arabic{section}}
\renewcommand\thesubsection{\thesection.\arabic{subsection}}
\renewcommand\thesubsubsection{\thesubsection.\arabic{subsubsection}}

\renewcommand\thesectiondis{\arabic{section}}
\renewcommand\thesubsectiondis{\thesectiondis.\arabic{subsection}}
\renewcommand\thesubsubsectiondis{\thesubsectiondis.\arabic{subsubsection}}

\hyphenation{op-tical net-works semi-conduc-tor}
\def\inputGnumericTable{}                                 

\lstset{
	%language=C,
	frame=single, 
	breaklines=true,
	columns=fullflexible
}
\lstset{
	%language=TeX,
	frame=single, 
	breaklines=true
}

\begin{document}
	
	
	\newtheorem{theorem}{Theorem}[section]
	\newtheorem{problem}{Problem}
	\newtheorem{proposition}{Proposition}[section]
	\newtheorem{lemma}{Lemma}[section]
	\newtheorem{corollary}[theorem]{Corollary}
	\newtheorem{example}{Example}[section]
	\newtheorem{definition}[problem]{Definition}
	
	\newcommand{\BEQA}{\begin{eqnarray}}
		\newcommand{\EEQA}{\end{eqnarray}}
	\newcommand{\define}{\stackrel{\triangle}{=}}
	\bibliographystyle{IEEEtran}
	\providecommand{\mbf}{\mathbf}
	\providecommand{\pr}[1]{\ensuremath{\Pr\left(#1\right)}}
	\providecommand{\qfunc}[1]{\ensuremath{Q\left(#1\right)}}
	\providecommand{\sbrak}[1]{\ensuremath{{}\left[#1\right]}}
	\providecommand{\lsbrak}[1]{\ensuremath{{}\left[#1\right.}}
	\providecommand{\rsbrak}[1]{\ensuremath{{}\left.#1\right]}}
	\providecommand{\brak}[1]{\ensuremath{\left(#1\right)}}
	\providecommand{\lbrak}[1]{\ensuremath{\left(#1\right.}}
	\providecommand{\rbrak}[1]{\ensuremath{\left.#1\right)}}
	\providecommand{\cbrak}[1]{\ensuremath{\left\{#1\right\}}}
	\providecommand{\lcbrak}[1]{\ensuremath{\left\{#1\right.}}
	\providecommand{\rcbrak}[1]{\ensuremath{\left.#1\right\}}}
	\theoremstyle{remark}
	\newtheorem{rem}{Remark}
	\newcommand{\sgn}{\mathop{\mathrm{sgn}}}
	\providecommand{\abs}[1]{\(\left\vert#1\right\vert\)}
	\providecommand{\res}[1]{\Res\displaylimits_{#1}} 
	\providecommand{\norm}[1]{\(\left\lVert#1\right\rVert\)}
	%\providecommand{\norm}[1]{\lVert#1\rVert}
	\providecommand{\mtx}[1]{\mathbf{#1}}
	\providecommand{\mean}[1]{E\(\left[ #1 \right]\)}
	\providecommand{\fourier}{\overset{\mathcal{F}}{ \rightleftharpoons}}
	%\providecommand{\hilbert}{\overset{\mathcal{H}}{ \rightleftharpoons}}
	\providecommand{\system}{\overset{\mathcal{H}}{ \longleftrightarrow}}
	%\newcommand{\solution}[2]{\textbf{Solution:}{#1}}
	\newcommand{\solution}{\noindent \textbf{Solution: }}
	\newcommand{\cosec}{\,\text{cosec}\,}
	\providecommand{\dec}[2]{\ensuremath{\overset{#1}{\underset{#2}{\gtrless}}}}
	\newcommand{\myvec}[1]{\ensuremath{\begin{psmallmatrix}#1\end{psmallmatrix}}}
	\newcommand{\mydet}[1]{\ensuremath{\begin{vmatrix}#1\end{vmatrix}}}
	%\numberwithin{equation}{section}
	\numberwithin{equation}{subsection}
	%\numberwithin{problem}{section}
	%\numberwithin{definition}{section}
	\makeatletter
	\@addtoreset{figure}{problem}
	\makeatother
	\let\StandardTheFigure\thefigure
	\let\vec\mathbf
	%\renewcommand{\thefigure}{\theproblem.\arabic{figure}}
	\renewcommand{\thefigure}{\theproblem}
	%\setlist[enumerate,1]{before=\renewcommand\theequation{\theenumi.\arabic{equation}}
	%\counterwithin{equation}{enumi}
	%\renewcommand{\theequation}{\arabic{subsection}.\arabic{equation}}
	\def\putbox#1#2#3{\makebox[0in][l]{\makebox[#1][l]{}\raisebox{\baselineskip}[0in][0in]{\raisebox{#2}[0in][0in]{#3}}}}
	\def\rightbox#1{\makebox[0in][r]{#1}}
	\def\centbox#1{\makebox[0in]{#1}}
	\def\topbox#1{\raisebox{-\baselineskip}[0in][0in]{#1}}
	\def\midbox#1{\raisebox{-0.5\baselineskip}[0in][0in]{#1}}
	\vspace{3cm}
	\title{Assignment 9}
	\author{Addagalla Satyanarayana}
	\maketitle
	\newpage
	%\tableofcontents
	\bigskip
	\renewcommand{\thefigure}{\theenumi}
	\renewcommand{\thetable}{\theenumi}
\begin{abstract}
This document shows the concept of markov chain state and tranistion matrices
\end{abstract}

%
\begin{lstlisting}
https://github.com/AddagallaSatyanarayana/AI5106/tree/master/Assignment9/Assignment9.tex
\end{lstlisting}
%
\section{Problem}
	Consider a Markov chain with state space {0,1,2,3,4} ad transition matrix 
\begin{align}
\vec{P}&=\myvec{1&0&0&0&0\\\frac{1}{3}&\frac{1}{3}&\frac{1}{3}&0&0\\0 &\frac{1}{3}&\frac{1}{3}&\frac{1}{3}&0\\0&0&\frac{1}{3}&\frac{1}{3}&\frac{1}{3}\\0&0&0&0&1}\label{eq:Problem}
\end{align}
Then find $\lim_{n \to \infty} p_{23}^{(n)}$
\section{Explanation}
 The probability of the transition from one state to another state in n steps can be calculated by computing the transition matrix raised to power n, ie $\vec{P}^n$.\newline
A matrix $\vec{P}$ can be written in the diagonalized  form if there exists an invertible matrix $\vec{S}$ such that
\begin{align}
\vec{P} = \vec{S}\vec{D}\vec{S}^{-1}\label{eq:1}
\end{align}
The matrix $\vec{S}$ is equal to the eigen vectors of matrix $\vec{P}$,where eigen values can be calculated using
\begin{align}
	\mydet{\vec{P}-\lambda\vec{I}}=0\label{eq:2}
\end{align}
The nth power of the matrix $\vec{P}$ can be calculated using 
\begin{align}
	\vec{P}^n = \vec{S}\vec{D}^n\vec{S}^{-1}\label{eq:nthpower}
\end{align} 

\section{Solution}

The eigen values for the matrix $\vec{P}$ can be found using \eqref{eq:2}
\begin{align}
 \mydet{1-\lambda&0&0&0&0\\\frac{1}{3}&\frac{1}{3}-\lambda&\frac{1}{3}&0&0\\0 &\frac{1}{3}&\frac{1}{3}-\lambda&\frac{1}{3}&0\\0&0&\frac{1}{3}&\frac{1}{3}-\lambda&\frac{1}{3}\\0&0&0&0&1-\lambda}=0
\end{align}
Solving this we get the eigen values of $\vec{P}$ as, 
\begin{align}
	\lambda_1 = 1\\
	\lambda_2=  \frac{1}{3}\\
	\lambda_3=  \frac{-\sqrt{2}+1}{3}\\
	\lambda_4=  \frac{\sqrt{2}+1}{3}
\end{align}
The corresponding eigen vectors are:
\begin{align}
	\vec v_1 &= \myvec{4& 3&2&1&0}^T\\
	\vec v_2 &= \myvec{-3& -2&-1&0&1}^T\\
	\vec v_3 &= \myvec{0& -1&0&1&0}^T\\
	\vec v_4 &= \myvec{0& 1&-\sqrt{2}&1&0}^T\\
	\vec v_5 &= \myvec{0& 1&\sqrt{2}&1&0}^T
\end{align}
The matrix $\vec{S}$ is equal to
\begin{align}
	\vec{S}=\myvec{
		4 & -3 & 0 & 0 & 0 \\
		3 & -2 & -1 & 1 & 1 \\
		2 & -1 & 0 & -\sqrt{2} & \sqrt{2} \\
		1 & 0 & 1 & 1 & 1 \\
		0 & 1 & 0 & 0 & 0
	}\label{eq:S_matrix}
\end{align}
The diagonal matrix $\vec{D}$ can be expressed using the eigen values in the form:
\begin{align}
	\vec{D} &=\myvec{
		1 & 0 & 0 & 0 & 0 \\
		0 & 1 & 0 & 0 & 0 \\
		0 & 0 & \frac{1}{3} & 0 & 0 \\
		0 & 0 & 0 & \frac{-\sqrt{2}+1}{3} & 0 \\
		0 & 0 & 0 & 0 & \frac{\sqrt{2}+1}{3}
	}\label{eq:D_matrix}
\end{align}
The matrix $\vec{S}^{-1}$ can be calculated using \eqref{eq:S_matrix}
\begin{align}
	\vec{S}^{-1}=\myvec{
		\frac{1}{4} & 0 & 0 & 0 & \frac{3}{4} \\
		0 & 0 & 0 & 0 & 1 \\
		\frac{1}{4} & \frac{-1}{2} & 0 & \frac{1}{2} & \frac{-1}{4} \\
		\frac{\sqrt{2}-2}{8} & \frac{1}{4} & \frac{-\sqrt{2}}{4} & \frac{1}{4} & \frac{\sqrt{2}-2}{8} \\
		\frac{-\sqrt{2}-2}{8} & \frac{1}{4} & \frac{\sqrt{2}}{4} & \frac{1}{4} & \frac{-\sqrt{2}-2}{8}
	}
\end{align}

Using equation \eqref{eq:nthpower}
\begin{align}
	\vec{P}^n&=\vec{S}\vec{D}^n\vec{S}^{-1}\\
	\vec{P}^n&=\myvec{
		4 & -3 & 0 & 0 & 0 \\
		3 & -2 & -1 & 1 & 1 \\
		2 & -1 & 0 & -\sqrt{2} & \sqrt{2} \\
		1 & 0 & 1 & 1 & 1 \\
		0 & 1 & 0 & 0 & 0
	}\myvec{
	1 & 0 & 0 & 0 & 0 \\
	0 & 1 & 0 & 0 & 0 \\
	0 & 0 & {(\frac{1}{3})}^n & 0 & 0 \\
	0 & 0 & 0 & {(\frac{-\sqrt{2}+1}{3})}^n & 0 \\
	0 & 0 & 0 & 0 & {(\frac{\sqrt{2}+1}{3})}^n
}\myvec{
\frac{1}{4} & 0 & 0 & 0 & \frac{3}{4} \\
0 & 0 & 0 & 0 & 1 \\
\frac{1}{4} & \frac{-1}{2} & 0 & \frac{1}{2} & \frac{-1}{4} \\
\frac{\sqrt{2}-2}{8} & \frac{1}{4} & \frac{-\sqrt{2}}{4} & \frac{1}{4} & \frac{\sqrt{2}-2}{8} \\
\frac{-\sqrt{2}-2}{8} & \frac{1}{4} & \frac{\sqrt{2}}{4} & \frac{1}{4} & \frac{-\sqrt{2}-2}{8}
}\label{eq:powern}	
\end{align}
Using \eqref{eq:powern},it can be shown that
\begin{align}
	\lim_{n \to \infty} p_{23}^{(n)}=0
\end{align}









	
\end{document}