\documentclass[journal,12pt,twocolumn]{IEEEtran}
%
\usepackage{setspace}
\usepackage{gensymb}
%\doublespacing
\singlespacing

\usepackage{graphicx}
\usepackage[cmex10]{amsmath}
\usepackage{amsmath,amsthm}
\usepackage{mathrsfs}
\usepackage{txfonts}
\usepackage{stfloats}
\usepackage{bm}
\usepackage{cite}
\usepackage{cases}
\usepackage{subfig}

\usepackage{longtable}
\usepackage{multirow}
\usepackage{commath}
\usepackage{enumitem}
\usepackage{mathtools}
\usepackage{steinmetz}
\usepackage{tikz}
\usepackage{circuitikz}
\usepackage{verbatim}
\usepackage{tfrupee}
\usepackage[breaklinks=true]{hyperref}

\usepackage{tkz-euclide}

\usetikzlibrary{calc,math}
\usepackage{listings}
\usepackage{color}                                            
\usepackage{array}                                            
\usepackage{longtable}                                        
\usepackage{calc}                                             
\usepackage{multirow}                                         
\usepackage{hhline}                                           
\usepackage{ifthen}                                           
\usepackage{lscape}     
\usepackage{multicol}
\usepackage{chngcntr}

\DeclareMathOperator*{\Res}{Res}

\renewcommand\thesection{\arabic{section}}
\renewcommand\thesubsection{\thesection.\arabic{subsection}}
\renewcommand\thesubsubsection{\thesubsection.\arabic{subsubsection}}

\renewcommand\thesectiondis{\arabic{section}}
\renewcommand\thesubsectiondis{\thesectiondis.\arabic{subsection}}
\renewcommand\thesubsubsectiondis{\thesubsectiondis.\arabic{subsubsection}}

\hyphenation{op-tical net-works semi-conduc-tor}
\def\inputGnumericTable{}                                 

\lstset{
	%language=C,
	frame=single, 
	breaklines=true,
	columns=fullflexible
}
\lstset{
	%language=TeX,
	frame=single, 
	breaklines=true
}

\begin{document}
	
	
	\newtheorem{theorem}{Theorem}[section]
	\newtheorem{problem}{Problem}
	\newtheorem{proposition}{Proposition}[section]
	\newtheorem{lemma}{Lemma}[section]
	\newtheorem{corollary}[theorem]{Corollary}
	\newtheorem{example}{Example}[section]
	\newtheorem{definition}[problem]{Definition}
	
	\newcommand{\BEQA}{\begin{eqnarray}}
		\newcommand{\EEQA}{\end{eqnarray}}
	\newcommand{\define}{\stackrel{\triangle}{=}}
	\bibliographystyle{IEEEtran}
	\providecommand{\mbf}{\mathbf}
	\providecommand{\pr}[1]{\ensuremath{\Pr\left(#1\right)}}
	\providecommand{\qfunc}[1]{\ensuremath{Q\left(#1\right)}}
	\providecommand{\sbrak}[1]{\ensuremath{{}\left[#1\right]}}
	\providecommand{\lsbrak}[1]{\ensuremath{{}\left[#1\right.}}
	\providecommand{\rsbrak}[1]{\ensuremath{{}\left.#1\right]}}
	\providecommand{\brak}[1]{\ensuremath{\left(#1\right)}}
	\providecommand{\lbrak}[1]{\ensuremath{\left(#1\right.}}
	\providecommand{\rbrak}[1]{\ensuremath{\left.#1\right)}}
	\providecommand{\cbrak}[1]{\ensuremath{\left\{#1\right\}}}
	\providecommand{\lcbrak}[1]{\ensuremath{\left\{#1\right.}}
	\providecommand{\rcbrak}[1]{\ensuremath{\left.#1\right\}}}
	\theoremstyle{remark}
	\newtheorem{rem}{Remark}
	\newcommand{\sgn}{\mathop{\mathrm{sgn}}}
	\providecommand{\abs}[1]{\(\left\vert#1\right\vert\)}
	\providecommand{\res}[1]{\Res\displaylimits_{#1}} 
	\providecommand{\norm}[1]{\(\left\lVert#1\right\rVert\)}
	%\providecommand{\norm}[1]{\lVert#1\rVert}
	\providecommand{\mtx}[1]{\mathbf{#1}}
	\providecommand{\mean}[1]{E\(\left[ #1 \right]\)}
	\providecommand{\fourier}{\overset{\mathcal{F}}{ \rightleftharpoons}}
	%\providecommand{\hilbert}{\overset{\mathcal{H}}{ \rightleftharpoons}}
	\providecommand{\system}{\overset{\mathcal{H}}{ \longleftrightarrow}}
	%\newcommand{\solution}[2]{\textbf{Solution:}{#1}}
	\newcommand{\solution}{\noindent \textbf{Solution: }}
	\newcommand{\cosec}{\,\text{cosec}\,}
	\providecommand{\dec}[2]{\ensuremath{\overset{#1}{\underset{#2}{\gtrless}}}}
	\newcommand{\myvec}[1]{\ensuremath{\begin{psmallmatrix}#1\end{psmallmatrix}}}
	\newcommand{\mydet}[1]{\ensuremath{\begin{vmatrix}#1\end{vmatrix}}}
	%\numberwithin{equation}{section}
	\numberwithin{equation}{subsection}
	%\numberwithin{problem}{section}
	%\numberwithin{definition}{section}
	\makeatletter
	\@addtoreset{figure}{problem}
	\makeatother
	\let\StandardTheFigure\thefigure
	\let\vec\mathbf
	%\renewcommand{\thefigure}{\theproblem.\arabic{figure}}
	\renewcommand{\thefigure}{\theproblem}
	%\setlist[enumerate,1]{before=\renewcommand\theequation{\theenumi.\arabic{equation}}
	%\counterwithin{equation}{enumi}
	%\renewcommand{\theequation}{\arabic{subsection}.\arabic{equation}}
	\def\putbox#1#2#3{\makebox[0in][l]{\makebox[#1][l]{}\raisebox{\baselineskip}[0in][0in]{\raisebox{#2}[0in][0in]{#3}}}}
	\def\rightbox#1{\makebox[0in][r]{#1}}
	\def\centbox#1{\makebox[0in]{#1}}
	\def\topbox#1{\raisebox{-\baselineskip}[0in][0in]{#1}}
	\def\midbox#1{\raisebox{-0.5\baselineskip}[0in][0in]{#1}}
	\vspace{3cm}
	\title{Assignment 9}
	\author{Addagalla Satyanarayana}
	\maketitle
	\newpage
	%\tableofcontents
	\bigskip
	\renewcommand{\thefigure}{\theenumi}
	\renewcommand{\thetable}{\theenumi}
\begin{abstract}
This document shows the concept of markov chain state and tranistion matrices
\end{abstract}

%
\begin{lstlisting}
https://github.com/AddagallaSatyanarayana/AI5106/tree/master/Assignment9/Assignment9.tex
\end{lstlisting}
%
\section{Problem}
	Consider a Markov chain with state space \{0,1,2,3,4\} and transition matrix 
\begin{align}
\vec{P}&=\myvec{
	1 & 0 & 0 & 0 & 0 \\
	\frac{1}{3} & \frac{1}{3} & \frac{1}{3} & 0 & 0 \\
	0 & \frac{1}{3} & \frac{1}{3} & \frac{1}{3} & 0 \\
	0 & 0 & \frac{1}{3} & \frac{1}{3} & \frac{1}{3} \\
	0 & 0 & 0 & 0 & 1
}\label{eq:Problem}
\end{align}
Draw the Markov chain and obtain the stationary probabilities
\section{Explanation}
 The Markov chain can be constructed from the transition matrix $\vec{P}$ where the states represent the nodes of  directed graph and the weights of the edges represent the transition probabilities. 
For a Markov chain with transition matrix P,a vector $\vec{v}$  is called a stationary distribution for $\vec{P}$ if and only if
\begin{align}
\vec{v}\vec{P}= \vec{v}\label{eq:stationary}
\end{align}
\section{Solution}
The Markov chain state can be drawn as shown below.
\begin{figure}[!ht]
	\centering
	\resizebox{\columnwidth}{!}{\usetikzlibrary{automata,positioning}
\begin{tikzpicture}

    % Add the states
    \node[state]             (s0) {0};
    \node[state, right=of s0] (s1) {1};
    \node[state, right=of s1] (s2) {2};
    \node[state, right=of s2] (s3) {3};
    \node[state, right=of s3] (s4) {4};
    % Connect the states with arrows
    \draw[every loop,auto=right,line width=1mm,>=latex,draw=orange, fill=orange]
      	(s0) edge[loop above]             node {1} (s0)
        (s1) edge[bend right, auto=right] node {1/3} (s0)
        (s1) edge[loop above]             node {1/3} (s1)
        (s1) edge[bend right, auto=right] node {1/3} (s2)
        (s2) edge[bend right, auto=right] node {1/3} (s1)
        (s2) edge[loop above]             node {1/3} (s2)
        (s2) edge[bend right, auto=right] node {1/3} (s3)
        (s3) edge[bend right, auto=right] node {1/3} (s2)
        (s3) edge[loop above]             node {1/3} (s3)
        (s3) edge[bend right, auto=right] node {1/3} (s4)
        (s4) edge[loop above]             node {1} (s4);
        
\end{tikzpicture}
}
	\caption{{Markov state diagram}}
	\label{fig1:State Diagram}
\end{figure}
The markov chain has multiple communicating classes namely \{0\} , \{4\} which are recurrent(ie remain in the same state) and \{1,2,3\} which is transient.\linebreak
It can observed that the equation \eqref{eq:stationary} is similar to the equation $\vec{v}\vec{P}= \lambda\vec{v}$ for eigenvalues and eigenvectors, with $\lambda = 1$.\linebreak
Taking the transpose of \eqref{eq:stationary} we get
\begin{align}
	{(\vec{v}\vec{P})}^T= \vec{v}^T\\
	\vec{P}^T\vec{v}^T=\vec{v}^T
\end{align}

The eigenvectors corresponding to the  eigenvalue $\lambda = 1$ of the transposed transition matrix $\vec{P}^T$ are the  stationary distributions of the chain.
When there are multiple eigenvectors associated to an eigenvalue of $\lambda = 1$, each such eigenvector is the  stationary distribution. 
\begin{align}
	\vec{P}&=\myvec{
			1 & 0 & 0 & 0 & 0 \\
			\frac{1}{3} & \frac{1}{3} & \frac{1}{3} & 0 & 0 \\
			0 & \frac{1}{3} & \frac{1}{3} & \frac{1}{3} & 0 \\
			0 & 0 & \frac{1}{3} & \frac{1}{3} & \frac{1}{3} \\
			0 & 0 & 0 & 0 & 1
		}\\
	\vec{P}^T&=\myvec{
		1 & \frac{1}{3} & 0 & 0 & 0 \\
		0 & \frac{1}{3} & \frac{1}{3} & 0 & 0 \\
		0 & \frac{1}{3} & \frac{1}{3} & \frac{1}{3} & 0 \\
		0 & 0 & \frac{1}{3} & \frac{1}{3} & 0 \\
		0 & 0 & 0 & \frac{1}{3} & 1}\label{eq:PTranspose}
\end{align}
The eigen values can be calculated using 
\begin{align}
	\mydet{\vec{P}^T-\lambda\vec{I}}=0\\
	\mydet{
		1-\lambda & \frac{1}{3} & 0 & 0 & 0 \\
		0 & \frac{1}{3}-\lambda & \frac{1}{3} & 0 & 0 \\
		0 & \frac{1}{3} & \frac{1}{3}-\lambda & \frac{1}{3} & 0 \\
		0 & 0 & \frac{1}{3} & \frac{1}{3}-\lambda & 0 \\
		0 & 0 & 0 & \frac{1}{3} & 1-\lambda} =0
\end{align} 
The eigen vectors corresponding to the eigen value $\lambda = 1$ are  
\begin{align}
	\vec v_1 &=\myvec{1 \\0 \\0 \\0 \\0	}\label{eq:stationary state1}\\
	\vec v_2 &=\myvec{0 \\0 \\0 \\0 \\1	}\label{eq:stationary state2}
\end{align}
The \eqref{eq:stationary state1} and \eqref{eq:stationary state2} represent the stationary probabilities.
\end{document}