\documentclass[journal,12pt,twocolumn]{IEEEtran}
%
\usepackage{setspace}
\usepackage{gensymb}
%\doublespacing
\singlespacing

\usepackage{graphicx}
\usepackage[cmex10]{amsmath}
\usepackage{amsmath,amsthm}
\usepackage{mathrsfs}
\usepackage{txfonts}
\usepackage{stfloats}
\usepackage{bm}
\usepackage{cite}
\usepackage{cases}
\usepackage{subfig}

\usepackage{longtable}
\usepackage{multirow}
\usepackage{commath}
\usepackage{enumitem}
\usepackage{mathtools}
\usepackage{steinmetz}
\usepackage{tikz}
\usepackage{circuitikz}
\usepackage{verbatim}
\usepackage{tfrupee}
\usepackage[breaklinks=true]{hyperref}

\usepackage{tkz-euclide}

\usetikzlibrary{calc,math}
\usepackage{listings}
\usepackage{color}                                            
\usepackage{array}                                            
\usepackage{longtable}                                        
\usepackage{calc}                                             
\usepackage{multirow}                                         
\usepackage{hhline}                                           
\usepackage{ifthen}                                           
\usepackage{lscape}     
\usepackage{multicol}
\usepackage{chngcntr}

\DeclareMathOperator*{\Res}{Res}

\renewcommand\thesection{\arabic{section}}
\renewcommand\thesubsection{\thesection.\arabic{subsection}}
\renewcommand\thesubsubsection{\thesubsection.\arabic{subsubsection}}

\renewcommand\thesectiondis{\arabic{section}}
\renewcommand\thesubsectiondis{\thesectiondis.\arabic{subsection}}
\renewcommand\thesubsubsectiondis{\thesubsectiondis.\arabic{subsubsection}}

\hyphenation{op-tical net-works semi-conduc-tor}
\def\inputGnumericTable{}                                 

\lstset{
	%language=C,
	frame=single, 
	breaklines=true,
	columns=fullflexible
}
\lstset{
	%language=TeX,
	frame=single, 
	breaklines=true
}

\begin{document}
	
	
	\newtheorem{theorem}{Theorem}[section]
	\newtheorem{problem}{Problem}
	\newtheorem{proposition}{Proposition}[section]
	\newtheorem{lemma}{Lemma}[section]
	\newtheorem{corollary}[theorem]{Corollary}
	\newtheorem{example}{Example}[section]
	\newtheorem{definition}[problem]{Definition}
	
	\newcommand{\BEQA}{\begin{eqnarray}}
		\newcommand{\EEQA}{\end{eqnarray}}
	\newcommand{\define}{\stackrel{\triangle}{=}}
	\bibliographystyle{IEEEtran}
	\providecommand{\mbf}{\mathbf}
	\providecommand{\pr}[1]{\ensuremath{\Pr\left(#1\right)}}
	\providecommand{\qfunc}[1]{\ensuremath{Q\left(#1\right)}}
	\providecommand{\sbrak}[1]{\ensuremath{{}\left[#1\right]}}
	\providecommand{\lsbrak}[1]{\ensuremath{{}\left[#1\right.}}
	\providecommand{\rsbrak}[1]{\ensuremath{{}\left.#1\right]}}
	\providecommand{\brak}[1]{\ensuremath{\left(#1\right)}}
	\providecommand{\lbrak}[1]{\ensuremath{\left(#1\right.}}
	\providecommand{\rbrak}[1]{\ensuremath{\left.#1\right)}}
	\providecommand{\cbrak}[1]{\ensuremath{\left\{#1\right\}}}
	\providecommand{\lcbrak}[1]{\ensuremath{\left\{#1\right.}}
	\providecommand{\rcbrak}[1]{\ensuremath{\left.#1\right\}}}
	\theoremstyle{remark}
	\newtheorem{rem}{Remark}
	\newcommand{\sgn}{\mathop{\mathrm{sgn}}}
	\providecommand{\abs}[1]{\(\left\vert#1\right\vert\)}
	\providecommand{\res}[1]{\Res\displaylimits_{#1}} 
	\providecommand{\norm}[1]{\(\left\lVert#1\right\rVert\)}
	%\providecommand{\norm}[1]{\lVert#1\rVert}
	\providecommand{\mtx}[1]{\mathbf{#1}}
	\providecommand{\mean}[1]{E\(\left[ #1 \right]\)}
	\providecommand{\fourier}{\overset{\mathcal{F}}{ \rightleftharpoons}}
	%\providecommand{\hilbert}{\overset{\mathcal{H}}{ \rightleftharpoons}}
	\providecommand{\system}{\overset{\mathcal{H}}{ \longleftrightarrow}}
	%\newcommand{\solution}[2]{\textbf{Solution:}{#1}}
	\newcommand{\solution}{\noindent \textbf{Solution: }}
	\newcommand{\cosec}{\,\text{cosec}\,}
	\providecommand{\dec}[2]{\ensuremath{\overset{#1}{\underset{#2}{\gtrless}}}}
	\newcommand{\myvec}[1]{\ensuremath{\begin{psmallmatrix}#1\end{psmallmatrix}}}
	\newcommand{\mydet}[1]{\ensuremath{\begin{vmatrix}#1\end{vmatrix}}}
	%\numberwithin{equation}{section}
	\numberwithin{equation}{subsection}
	%\numberwithin{problem}{section}
	%\numberwithin{definition}{section}
	\makeatletter
	\@addtoreset{figure}{problem}
	\makeatother
	\let\StandardTheFigure\thefigure
	\let\vec\mathbf
	%\renewcommand{\thefigure}{\theproblem.\arabic{figure}}
	\renewcommand{\thefigure}{\theproblem}
	%\setlist[enumerate,1]{before=\renewcommand\theequation{\theenumi.\arabic{equation}}
	%\counterwithin{equation}{enumi}
	%\renewcommand{\theequation}{\arabic{subsection}.\arabic{equation}}
	\def\putbox#1#2#3{\makebox[0in][l]{\makebox[#1][l]{}\raisebox{\baselineskip}[0in][0in]{\raisebox{#2}[0in][0in]{#3}}}}
	\def\rightbox#1{\makebox[0in][r]{#1}}
	\def\centbox#1{\makebox[0in]{#1}}
	\def\topbox#1{\raisebox{-\baselineskip}[0in][0in]{#1}}
	\def\midbox#1{\raisebox{-0.5\baselineskip}[0in][0in]{#1}}
	\vspace{3cm}
	\title{Assignment 7}
	\author{Addagalla Satyanarayana}
	\maketitle
	\newpage
	%\tableofcontents
	\bigskip
	\renewcommand{\thefigure}{\theenumi}
	\renewcommand{\thetable}{\theenumi}
\begin{abstract}
This document uses the properties of a parabola
\end{abstract}
Download latex-tikz codes from 
%
\begin{lstlisting}
https://github.com/AddagallaSatyanarayana/AI5106/tree/master/Assignment7/Assignment7.tex
\end{lstlisting}
%
\section{Problem}
	Trace the parabola
\begin{align}
	144x^2-120xy+25y^2+619x-272y+663=0\label{eq:1}
\end{align}

\section{Explanation}
The general equation of second degree is given by
\begin{align}
	ax^2+2bxy+cy^2+2dx+2ey+f=0 \label{gen_quad_eqn}
\end{align}
and can be expressed as
\begin{align}
	\vec{x}^T\vec{V}\vec{x}+2\vec{u}^T\vec{x}+f=0 \label{conic_quad_eqn}
\end{align}
where
\begin{align}
	\vec{V} &= \vec{V}^T = \myvec{a & b \\ b & c}	\\
	\vec{u}^T &= \myvec{d & e}
\end{align}

From equation \eqref{eq:1} , we get

\begin{align}
	\vec{V} &= \myvec{144&-60\\-60&25}\label{eq:2}\\
	\vec{u} &= \myvec{\frac{619}{2}\\-\frac{272}{2}}\label{eq:3}\\ 
	f &= 663 \label{eq:conics/ex/solution/given2}
\end{align}
Expanding the determinant of $\vec{V}$ we observe, 
\begin{align}
	\mydet{144&-60\\-60&25} = 0 \label{eq:4}
\end{align}
The characteristic equation of $\vec{V}$ is given as follows,
\begin{align}
		\mydet{\lambda\vec{I}-\vec{V}} = \mydet{\lambda-144&60\\60&\lambda-25} &= 0\\
		\implies \lambda^2-169\lambda &= 0\label{eq:5}
\end{align}
 the eigenvalues are given by
\begin{align}
		\lambda_1=0, \lambda_2=169\label{eq:6}    
\end{align}
For $\lambda_1 = 0$, the eigen vector $\vec{p}$ is given by 
\begin{align}
		\vec{V}\vec{p} = 0
\end{align}
Row reducing $\vec{V}$ 
\begin{align}
		\implies
		\myvec{-144&60\\60&-25}\xleftrightarrow[R_2=R_2+5R_1]{R_1=\frac{R_1}{12}}\myvec{-12&5\\0&0}\\
		\implies\vec{p}_1=\frac{1}{13}\myvec{5\\12} \label{eq:7}
\end{align}
Similarly, 
\begin{align}
		\vec{p}_2=\frac{1}{13}\myvec{12\\-5} \label{eq:8}
\end{align}
	Thus,
\begin{align}
		\vec{P}&=\myvec{\vec{p_1}&\vec{p_2}}=\frac{1}{13}\myvec{5&12\\ 12 &-5} 
\end{align}
and its equation is
\begin{align}
		\vec{y^T}\vec{D}\vec{y}&=-2\eta\myvec{1&0}\vec{y}
\end{align}
where
\begin{align}
		\eta=\vec{u}^T\vec{p_1}=-6.5
\end{align}
\subsection{Finding QR decompostion of V}
%QR DECOMPOSITION
The QR decomposition of
$\vec{V}$ can be written as,
\begin{align}
	\vec{V} = \myvec{\vec{a} & \vec{b}} \label{eq:10}
\end{align}
where $\vec{a}$ and $\vec{b}$ and  are column vectors,
\begin{align}
	\vec{a} = \myvec{144 \\ -60} \label{eq:11}
\end{align}
\begin{align}
	\vec{b} = \myvec{-60 \\25} \label{eq:12}
\end{align}
\begin{align}
	\vec{V} = \vec{QR}\label{eq:13}
\end{align}
where $\vec{R}$ is a upper triangular matrix and $\vec{Q}$ such that, 
\begin{align}
	\vec{Q^TQ} = \vec{I}\label{eq:14}
\end{align}
and 
\begin{align}
	\vec{Q} = \myvec{\vec{q_1} & \vec{q_2}} \ and \ \vec{R} = \myvec{r_1 & r_2 \\ 0 & r_3}  \label{eq:15}  
\end{align}
$\vec{q_1}$, $\vec{q_2}$ and the values in $\vec{R}$ are given by,
\begin{align}
	r_1 = \norm{\vec{a}}\\\label{eq:16}
	\vec{q_1} = \frac{\vec{a}}{r_1}\\\label{eq:17}
	r_2 = \frac{\vec{q_1^Tb}}{\norm{\vec{q_1}}^{2}}\\
	\vec{q_2} = \frac{\vec{b} - r_2\vec{q_1}}{\norm{\vec{b}-r_2\vec{q_1}}}\\
	r_3 = \vec{q_2^Tb}
\end{align}
Hence,
\begin{align}
	r_1 = \sqrt{24336}=156\\
	\vec{q_1} = \frac{1}{156}\myvec{144 \\ -60} = \myvec{\frac{12}{13} \\[1em] \frac{-5}{13}}\\
	r_2 = -65\\
	\vec{q_2} = \sqrt{5}\myvec{\frac{1}{5} \\[1em] \frac{2}{5}} = \myvec{0 \\0}\\
	r_3 = 0
\end{align}
Therefore,
\begin{align}
	\vec{QR} = \myvec{\frac{12}{13} & 0 \\ \frac{-5}{13} & 0}\myvec{156 & -65 \\ 0 & 0}\\
	\implies \vec{QR} = \myvec{144 & -60 \\ -60 & 25} \label{eq:19}
\end{align}
As \eqref{eq:2} and \eqref{eq:19} are equal, the QR decomposition holds.

\subsection{Finding vertex using SVD}
The equation of perpendicular line passing through focus and intersecting parabola at vertex $\vec c$ is given as
\begin{align}
	\myvec{\vec{u^T}+\frac{\eta}{2}\vec{p_1^T} \\ \vec{V}}\vec{c}=\myvec{-f \\ \frac{\eta}{2}\vec{p_1}-\vec{u}} 
\end{align}
\begin{align}
	&\myvec{\frac{1233}{4} & -139\\144 & -60\\-60 & 25}\vec c = \myvec{-663\\\frac{1243}{4}\\133}\\
	&\implies \vec M \vec c= \vec b\label{eq: Mc=b}
\end{align}
To solve \eqref{eq: Mc=b}, we perform Singular Value Decomposition on $\vec{M}$ given as 
\begin{align}
	\vec{M = USV^T }\label{eq: SVD}
\end{align}
Putting this value of $\vec{M}$ in \eqref{eq: Mc=b}, we get
\begin{align}
	&\vec{USV^T}\vec{c} = \vec{b} \\
	\implies& \vec{c} = \vec{VS_+U^T}\vec{b}\label{eq: c}
\end{align}
where, $\vec{S_+}$ is Moore-Penrose pseudo-inverse of $\vec{S}$. Columns of $\vec{U}$ are eigen-vectors of $\vec{MM^T}$, columns of $\vec{V}$ are eigen-vectors of $\vec{M^TM}$ and $\vec{S}$ is diagonal matrix of singular value of eigenvalues of $\vec{M^TM}$.
\begin{align}
	\vec {MM^T} &= \myvec{308 & -139\\144 & -60\\ -60 & 25} \myvec {308 & 144 & -60\\ -139 & -60 & 25}\\[1em]
	& = \myvec{114340 & 52728 & -21970\\ 52728 & 24336 & -10140\\-21970 & -10140 & 4225}\\[1em]
	\vec{M^TM}=& \myvec {308 & 144 & -60\\ -139 & -60 & 25} \myvec{308 & -139\\144 & -60\\ -60 & 25}\\[1em]
	&= \myvec{119354 & -52986\\ -52986 & 23546}
\end{align}
Eigen values of $\vec{M^TM}$ can be found out as
\begin{align}
	&\abs{\vec{M^TM-\lambda I}} = 0\\[1em]
	\implies & \mydet{119354- \lambda & -52986\\ -52986 & 23546 -\lambda}=0
\end{align}	
Solving this we get the eigen values of $\vec{M^TM}$ as, 
\begin{align}
	\lambda_1 = 19\\
	\lambda_2=  142880
\end{align}
The corresponding normalized eigen vectors are:
\begin{align}
	\vec v_1 = \myvec{-0.9140\\ 0.4058}\\
	\vec v_2 = \myvec{-0.4058\\ -0.9140}
\end{align}
Hence, 
\begin{align}
	\vec V=& \myvec{\vec v_1 & \vec v_2}
	=\myvec{-0.9140 & -0.4058\\0.4058& -0.9140}
\end{align}
Eigen values of $\vec{MM^T}$ can be found by solving:
\begin{align}
	&\abs{\vec{MM^T}-\lambda \vec {I}} = 0\\
	\implies & \mydet{114340- \lambda & 52728 & -21970\\ 52728 & 24336- \lambda & -10140\\ -21970 & -10140 & 4225-\lambda}=0
\end{align}
Solving this, we get the eigen values of $\vec{MM^T}$ as:
\begin{align}
	\lambda_3= 19\\
	\lambda_4= 142880\\
	\lambda_5= 0
\end{align}
The corresponding eigen vectors after normalizing are:
\begin{align}
	\vec u_1 = \myvec{-0.8945\\-0.4126\\0.1719}\\
	\vec u_2= \myvec{0.4470\\-0.8257\\0.3441}\\
	\vec u_3= \myvec{0 \\0.3846\\ 0.9231}\\
	\therefore \vec U= \myvec{-0.8945& 0.4470 & 0\\
		-0.4126 & -0.8257 & 0.3846\\
		0.1719 & 0.3441 & 0.9231}
\end{align}
After computing the singular values from the eigen values, \begin{align}
	\vec S &= \myvec{\sqrt{\lambda _1} & 0\\ 0 & \sqrt{\lambda_2}\\ 0 & 0}
	= \myvec {378 & 0 \\ 0 & 4.35 \\0 & 0}
\end{align}
Therefore we get the SVD of $\vec M $ as:
\begin{align}
	\vec M = \myvec{-0.8945& 0.4470 & 0\\
		-0.4126 & -0.8257 & 0.3846\\
		0.1719 & 0.3441 & 0.9231} \myvec {378 & 0 \\ 0 & 4.35 \\0 & 0}\\ \myvec{-0.9140 & -0.4058\\0.4058& -0.9140}^T
\end{align}
\begin{align}
	= \myvec{308 & -139\\144 & -60\\ -60 & 25} 
\end{align}
Moore- penrose pseudo inverse of $\vec S $ is:
\begin{align}
	\vec S_+ = \myvec{0.0026 &0 &0\\ 0 & 0.2294 & 0}
\end{align}
Putting the values in \eqref{eq: c},
\begin{align}
	\vec {U^Tb}= \myvec{744.132\\  5.99\\3.25}\\
	\vec {S_+ U^T b}= \myvec{1.964\\  1.3736}\\
	\vec {c}= \vec{S_+ U^T b}= \myvec{-2.35\\-0.458}\label{eq: svd sol c}
\end{align}

\subsection{Verification using least square method}
%Verification using least square method
\begin{align}
	\myvec{308 & -139\\144 & -60\\ -60 & 25}\vec{c}=\myvec{-663\\\frac{1243}{4}\\133}
\end{align}
This is in the form of 
\begin{align}
 {A}\vec{c}=\vec{b}
 \end{align}
  \begin{align} 
 {A}^T{A}\vec{c}={A}^T\vec{b}
 \implies \vec{c}=  ({{A}^T{A}})^{-1} {A}^T\vec{b} \label{eq:20}
\end{align}
\begin{align}
{A}^T{A}=& \myvec {308 & 144 & -60\\ -139 & -60 & 25} \myvec{308 & -139\\144 & -60\\ -60 & 25}\\[1em]
&= \myvec{119354 & -52986.8\\ -52986.8 & 23546}
\end{align}
The inverse can be written as
\begin{align}
    ({{A}^T{A}})^{-1}=
    \myvec{0.00867 & 0.01951\\0.01951 & 0.04395}\label{eq:21}
\end{align} 
\begin{align}
   {A}^T\vec{b} =\myvec{-257098 \\ 114127} \label{eq:22}
\end{align}
using \ref{eq:21} and \ref{eq:22} in \ref{eq:20},the center $\vec{c}$
\begin{align}
    \vec{c}=
    \myvec{0.00867 & 0.01951\\0.01951 & 0.04395}\myvec{-257098 \\ 114127} \\
    \implies \vec{c}=\myvec{-2.363 \\ -0.453}\label{eq:LS sol c}  
\end{align}
Comparing \eqref{eq: svd sol c} and \eqref{eq:LS sol c}, it can be said that the solution of $\vec c$ is verified.
\end{document}