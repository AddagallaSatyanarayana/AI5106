\documentclass[journal,12pt,twocolumn]{IEEEtran}
%
\usepackage{setspace}
\usepackage{gensymb}
%\doublespacing
\singlespacing

\usepackage{graphicx}
\usepackage[cmex10]{amsmath}
\usepackage{amsmath,amsthm}
\usepackage{mathrsfs}
\usepackage{txfonts}
\usepackage{stfloats}
\usepackage{bm}
\usepackage{cite}
\usepackage{cases}
\usepackage{subfig}

\usepackage{longtable}
\usepackage{multirow}
\usepackage{commath}
\usepackage{enumitem}
\usepackage{mathtools}
\usepackage{steinmetz}
\usepackage{tikz}
\usepackage{circuitikz}
\usepackage{verbatim}
\usepackage{tfrupee}
\usepackage[breaklinks=true]{hyperref}

\usepackage{tkz-euclide}

\usetikzlibrary{calc,math}
\usepackage{listings}
\usepackage{color}                                            
\usepackage{array}                                            
\usepackage{longtable}                                        
\usepackage{calc}                                             
\usepackage{multirow}                                         
\usepackage{hhline}                                           
\usepackage{ifthen}                                           
\usepackage{lscape}     
\usepackage{multicol}
\usepackage{chngcntr}

\DeclareMathOperator*{\Res}{Res}

\renewcommand\thesection{\arabic{section}}
\renewcommand\thesubsection{\thesection.\arabic{subsection}}
\renewcommand\thesubsubsection{\thesubsection.\arabic{subsubsection}}

\renewcommand\thesectiondis{\arabic{section}}
\renewcommand\thesubsectiondis{\thesectiondis.\arabic{subsection}}
\renewcommand\thesubsubsectiondis{\thesubsectiondis.\arabic{subsubsection}}

\hyphenation{op-tical net-works semi-conduc-tor}
\def\inputGnumericTable{}                                 

\lstset{
	%language=C,
	frame=single, 
	breaklines=true,
	columns=fullflexible
}
\lstset{
	%language=TeX,
	frame=single, 
	breaklines=true
}

\begin{document}
	
	
	\newtheorem{theorem}{Theorem}[section]
	\newtheorem{problem}{Problem}
	\newtheorem{proposition}{Proposition}[section]
	\newtheorem{lemma}{Lemma}[section]
	\newtheorem{corollary}[theorem]{Corollary}
	\newtheorem{example}{Example}[section]
	\newtheorem{definition}[problem]{Definition}
	
	\newcommand{\BEQA}{\begin{eqnarray}}
		\newcommand{\EEQA}{\end{eqnarray}}
	\newcommand{\define}{\stackrel{\triangle}{=}}
	\bibliographystyle{IEEEtran}
	\providecommand{\mbf}{\mathbf}
	\providecommand{\pr}[1]{\ensuremath{\Pr\left(#1\right)}}
	\providecommand{\qfunc}[1]{\ensuremath{Q\left(#1\right)}}
	\providecommand{\sbrak}[1]{\ensuremath{{}\left[#1\right]}}
	\providecommand{\lsbrak}[1]{\ensuremath{{}\left[#1\right.}}
	\providecommand{\rsbrak}[1]{\ensuremath{{}\left.#1\right]}}
	\providecommand{\brak}[1]{\ensuremath{\left(#1\right)}}
	\providecommand{\lbrak}[1]{\ensuremath{\left(#1\right.}}
	\providecommand{\rbrak}[1]{\ensuremath{\left.#1\right)}}
	\providecommand{\cbrak}[1]{\ensuremath{\left\{#1\right\}}}
	\providecommand{\lcbrak}[1]{\ensuremath{\left\{#1\right.}}
	\providecommand{\rcbrak}[1]{\ensuremath{\left.#1\right\}}}
	\theoremstyle{remark}
	\newtheorem{rem}{Remark}
	\newcommand{\sgn}{\mathop{\mathrm{sgn}}}
	\providecommand{\abs}[1]{\(\left\vert#1\right\vert\)}
	\providecommand{\res}[1]{\Res\displaylimits_{#1}} 
	\providecommand{\norm}[1]{\(\left\lVert#1\right\rVert\)}
	%\providecommand{\norm}[1]{\lVert#1\rVert}
	\providecommand{\mtx}[1]{\mathbf{#1}}
	\providecommand{\mean}[1]{E\(\left[ #1 \right]\)}
	\providecommand{\fourier}{\overset{\mathcal{F}}{ \rightleftharpoons}}
	%\providecommand{\hilbert}{\overset{\mathcal{H}}{ \rightleftharpoons}}
	\providecommand{\system}{\overset{\mathcal{H}}{ \longleftrightarrow}}
	%\newcommand{\solution}[2]{\textbf{Solution:}{#1}}
	\newcommand{\solution}{\noindent \textbf{Solution: }}
	\newcommand{\cosec}{\,\text{cosec}\,}
	\providecommand{\dec}[2]{\ensuremath{\overset{#1}{\underset{#2}{\gtrless}}}}
	\newcommand{\myvec}[1]{\ensuremath{\begin{psmallmatrix}#1\end{psmallmatrix}}}
	\newcommand{\mydet}[1]{\ensuremath{\begin{vmatrix}#1\end{vmatrix}}}
	%\numberwithin{equation}{section}
	\numberwithin{equation}{subsection}
	%\numberwithin{problem}{section}
	%\numberwithin{definition}{section}
	\makeatletter
	\@addtoreset{figure}{problem}
	\makeatother
	\let\StandardTheFigure\thefigure
	\let\vec\mathbf
	%\renewcommand{\thefigure}{\theproblem.\arabic{figure}}
	\renewcommand{\thefigure}{\theproblem}
	%\setlist[enumerate,1]{before=\renewcommand\theequation{\theenumi.\arabic{equation}}
	%\counterwithin{equation}{enumi}
	%\renewcommand{\theequation}{\arabic{subsection}.\arabic{equation}}
	\def\putbox#1#2#3{\makebox[0in][l]{\makebox[#1][l]{}\raisebox{\baselineskip}[0in][0in]{\raisebox{#2}[0in][0in]{#3}}}}
	\def\rightbox#1{\makebox[0in][r]{#1}}
	\def\centbox#1{\makebox[0in]{#1}}
	\def\topbox#1{\raisebox{-\baselineskip}[0in][0in]{#1}}
	\def\midbox#1{\raisebox{-0.5\baselineskip}[0in][0in]{#1}}
	\vspace{3cm}
	\title{Assignment 8}
	\author{Addagalla Satyanarayana}
	\maketitle
	\newpage
	%\tableofcontents
	\bigskip
	\renewcommand{\thefigure}{\theenumi}
	\renewcommand{\thetable}{\theenumi}
\begin{abstract}
This document shows the diagonalization of matrices
\end{abstract}

%
\begin{lstlisting}
https://github.com/AddagallaSatyanarayana/AI5106/tree/master/Assignment8/Assignment8.tex
\end{lstlisting}
%
\section{Problem}
	Which of the following matrices are not diagonalizable over R
\begin{align}
\vec{M_1}&=\myvec{2&0&1\\0&3&0\\0 & 0&2}\\
\vec{M_2}&=\myvec{1&1\\1&1}\\
\vec{M_3}&=\myvec{2&1&0\\0&3&0\\0 & 0&3}\\
\vec{M_4}&=\myvec{1&-1\\2&4}
\end{align}

\section{Explanation}
A matrix $\vec{M}$ can be diagonalized if there exists an invertible matrix $\vec{P}$ such that
\begin{align}
\vec{M} = \vec{P}\vec{D}\vec{P}^{-1}\label{eq:1}
\end{align}
The matrix $\vec{P}$ is equal to the eigen vectors of matrix $\vec{M}$,where eigen values can be calculated using
\begin{align}
	\mydet{\vec{M}-\lambda\vec{I}}=0\label{eq:2}
\end{align}
\subsection{Solving matrix1 }
\begin{align}
	\myvec{2&0&1\\0&3&0\\0 & 0&2}
\end{align}
The eigen values for the matrix $\vec{M}$ can be found using \eqref{eq:2}
\begin{align}
 \mydet{2- \lambda & 0 & 1\\ 0 & 3 -\lambda&0\\0& 0& 2-\lambda}=0
\end{align}
Solving this we get the eigen values of $\vec{M}$ as, 
\begin{align}
	\lambda_1 = 2\\
	\lambda_2=  3
\end{align}
The corresponding eigen vectors are:
\begin{align}
	\vec v_1 = \myvec{1\\ 0\\0};
	\vec v_2 = \myvec{0\\ 1\\0}
\end{align}
For diagonalization the number of eigen vectors should be equal to the matrix dimensions.
Since there are only 2 eigen vectors and the dimension of $\vec{M}$ is 3 , $\vec{M}$ is not diagonalizable.
\subsection{Solving matrix2}
\begin{align}
	\myvec{1&1\\1&1}\label{problem:2}
\end{align}
The eigen values for the matrix $\vec{M}$ can be found using \eqref{eq:2}
\begin{align}
	\mydet{1- \lambda &  1\\1& 1-\lambda}=0
\end{align}
Solving this we get the eigen values of $\vec{M}$ as, 
\begin{align}
	\lambda_1 = 0\\
	\lambda_2=  2
\end{align}
The corresponding eigen vectors are:
\begin{align}
	\vec v_1 = \myvec{-1\\ 1};
	\vec v_2 = \myvec{1\\ 1}
\end{align}
The matrix $\vec{P}$ is equal to
\begin{align}
	\vec{P}=\myvec{-1&1\\1&1}\label{eq:2_1}
\end{align}
The diagonal matrix $\vec{D}$ can be expressed using the eigen values in the form:
\begin{align}
	\vec{D}&=\myvec{\lambda_1&0\\0&\lambda_2}\\
		\vec{D} &=\myvec{0&0\\0&2}\label{eq:2_2}
\end{align}
The matrix $\vec{P}^{-1}$ can be calculated using \eqref{eq:2_2}
\begin{align}
	\vec{P}^{-1}=\myvec{-0.5&0.5\\0.5&0.5}\label{eq:2_3}
\end{align}
Using \eqref{eq:2_1},\eqref{eq:2_2} , \eqref{eq:2_3} and \eqref{eq:1}
\begin{align}
	\vec{P}\vec{D}\vec{P}^{-1}&=\myvec{-1&1\\1&1}\myvec{0&0\\0&2}\myvec{-0.5&0.5\\0.5&0.5}\\
	\vec{M}&=\myvec{1&1\\1&1}\label{solution:2}	
\end{align}
From equation \eqref{solution:2} and equation \eqref{problem:2} we get,$\vec{M} = \vec{P}\vec{D}\vec{P}^{-1}$.Hence the matrix is diagonalizable. 
\subsection{Solving matrix3 }
\begin{align}
	\myvec{2&1&0\\0&3&0\\0 & 0&3}\label{problem:3}
\end{align}
The eigen values for the matrix $\vec{M}$ can be found \eqref{eq:2}
\begin{align}
	\mydet{2- \lambda & 1 & 0\\ 0 & 3 -\lambda&0\\0& 0& 3-\lambda}=0
\end{align}
Solving this we get the eigen values of $\vec{M}$ as, 
\begin{align}
	\lambda_1 = 2\\
	\lambda_2=  3
\end{align}
The corresponding eigen vectors are:
\begin{align}
	\vec v_1 = \myvec{1\\ 0\\0};
	\vec v_2 = \myvec{1\\ 1\\0};
	\vec v_3 = \myvec{0\\ 0\\1}
\end{align}
The matrix $\vec{P}$ is equal to
\begin{align}
	\vec{P}=\myvec{1&1&0\\0&1&0\\0&0&1}\label{eq:3_1}
\end{align}
The diagonal matrix $\vec{D}$ can be expressed using the eigen values in the form:
\begin{align}
	\vec{D}&=\myvec{\lambda_1&0&0\\0&\lambda_2&0\\0&0&\lambda_3}\\
	\vec{D} &=\myvec{2&0&0\\0&3&0\\0&0&2}\label{eq:3_2}
\end{align}
The inverse matrix $\vec{P}^{-1}$ can be calculated using \eqref{eq:3_2}
\begin{align}
	\vec{P}^{-1}=\myvec{1&-1&0\\0&1&0\\0&0&1}\label{eq:3_3}
\end{align}
Using \eqref{eq:3_1},\eqref{eq:3_2} , \eqref{eq:3_3} and \eqref{eq:1}
\begin{align}
	\vec{P}\vec{D}\vec{P}^{-1}&=\myvec{1&1&0\\0&1&0\\0&0&1}\myvec{2&0&0\\0&3&0\\0&0&2}\myvec{1&-1&0\\0&1&0\\0&0&1}\\
	\vec{M}&=\myvec{2&1&0\\0&3&0\\0 & 0&3}\label{solution:3}	
\end{align}
From equation \eqref{solution:3} and equation \eqref{problem:3},$\vec{M} = \vec{P}\vec{D}\vec{P}^{-1}$.Hence the matrix is diagonalizable. 

\subsection{Solving matrix4 }
\begin{align}
	\myvec{1&-1\\2&4}\label{problem:4}
\end{align}
The eigen values for the matrix $\vec{M}$ can be found using \eqref{eq:2}
\begin{align}
	\mydet{1- \lambda &  -1\\2& 4-\lambda}=0
\end{align}
Solving this we get the eigen values of $\vec{M}$ as, 
\begin{align}
	\lambda_1 = 2\\
	\lambda_2=  3
\end{align}
The corresponding eigen vectors are:
\begin{align}
	\vec v_1 = \myvec{-1\\ 1};
	\vec v_2 = \myvec{-0.5\\ 1}
\end{align}
The matrix $\vec{P}$ is equal to
\begin{align}
	\vec{P}=\myvec{-1&-0.5\\1&1}\label{eq:4_1}
\end{align}
The diagonal matrix $\vec{D}$ can be expressed using the eigen values in the form:
\begin{align}
	\vec{D}&=\myvec{\lambda_1&0\\0&\lambda_2}\\
	\vec{D} &=\myvec{2&0\\0&3}\label{eq:4_2}
\end{align}
The inverse matrix $\vec{P}^{-1}$ can be calculated using \eqref{eq:4_2}
\begin{align}
	\vec{P}^{-1}=\myvec{-2&-1\\2&2}\label{eq:4_3}
\end{align}
Using \eqref{eq:4_1},\eqref{eq:4_2} , \eqref{eq:4_3} and \eqref{eq:1}
\begin{align}
	\vec{P}\vec{D}\vec{P}^{-1}&=\myvec{-1&-0.5\\1&1}\myvec{0&0\\0&2}\myvec{2&0\\0&3}\myvec{-2&-1\\2&2}\\
	\vec{M}&=\myvec{1&-1\\2&4}\label{solution:4}	
\end{align}
From equation \eqref{solution:4} and equation \eqref{problem:4},$\vec{M} = \vec{P}\vec{D}\vec{P}^{-1}$.Hence the matrix is diagonalizable. 
\subsection{Conclusion: }
\begin{align}
	\vec{M_1}&=\myvec{2&0&1\\0&3&0\\0 & 0&2}\nonumber
\end{align}
is the matrix which is not diagonalizable.

	
\end{document}