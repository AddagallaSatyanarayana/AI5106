\documentclass[journal,12pt,twocolumn]{IEEEtran}
%
\usepackage{setspace}
\usepackage{gensymb}
%\doublespacing
\singlespacing

\usepackage{graphicx}
\usepackage[cmex10]{amsmath}
\usepackage{amsmath,amsthm}
\usepackage{mathrsfs}
\usepackage{txfonts}
\usepackage{stfloats}
\usepackage{bm}
\usepackage{cite}
\usepackage{cases}
\usepackage{subfig}

\usepackage{longtable}
\usepackage{multirow}
\usepackage{commath}
\usepackage{enumitem}
\usepackage{mathtools}
\usepackage{steinmetz}
\usepackage{tikz}
\usepackage{circuitikz}
\usepackage{verbatim}
\usepackage{tfrupee}
\usepackage[breaklinks=true]{hyperref}

\usepackage{tkz-euclide}

\usetikzlibrary{calc,math}
\usepackage{listings}
\usepackage{color}                                            
\usepackage{array}                                            
\usepackage{longtable}                                        
\usepackage{calc}                                             
\usepackage{multirow}                                         
\usepackage{hhline}                                           
\usepackage{ifthen}                                           
\usepackage{lscape}     
\usepackage{multicol}
\usepackage{chngcntr}

\DeclareMathOperator*{\Res}{Res}

\renewcommand\thesection{\arabic{section}}
\renewcommand\thesubsection{\thesection.\arabic{subsection}}
\renewcommand\thesubsubsection{\thesubsection.\arabic{subsubsection}}

\renewcommand\thesectiondis{\arabic{section}}
\renewcommand\thesubsectiondis{\thesectiondis.\arabic{subsection}}
\renewcommand\thesubsubsectiondis{\thesubsectiondis.\arabic{subsubsection}}

\hyphenation{op-tical net-works semi-conduc-tor}
\def\inputGnumericTable{}                                 

\lstset{
	%language=C,
	frame=single, 
	breaklines=true,
	columns=fullflexible
}
\lstset{
	%language=TeX,
	frame=single, 
	breaklines=true
}

\begin{document}
	
	
	\newtheorem{theorem}{Theorem}[section]
	\newtheorem{problem}{Problem}
	\newtheorem{proposition}{Proposition}[section]
	\newtheorem{lemma}{Lemma}[section]
	\newtheorem{corollary}[theorem]{Corollary}
	\newtheorem{example}{Example}[section]
	\newtheorem{definition}[problem]{Definition}
	
	\newcommand{\BEQA}{\begin{eqnarray}}
		\newcommand{\EEQA}{\end{eqnarray}}
	\newcommand{\define}{\stackrel{\triangle}{=}}
	\bibliographystyle{IEEEtran}
	\providecommand{\mbf}{\mathbf}
	\providecommand{\pr}[1]{\ensuremath{\Pr\left(#1\right)}}
	\providecommand{\qfunc}[1]{\ensuremath{Q\left(#1\right)}}
	\providecommand{\sbrak}[1]{\ensuremath{{}\left[#1\right]}}
	\providecommand{\lsbrak}[1]{\ensuremath{{}\left[#1\right.}}
	\providecommand{\rsbrak}[1]{\ensuremath{{}\left.#1\right]}}
	\providecommand{\brak}[1]{\ensuremath{\left(#1\right)}}
	\providecommand{\lbrak}[1]{\ensuremath{\left(#1\right.}}
	\providecommand{\rbrak}[1]{\ensuremath{\left.#1\right)}}
	\providecommand{\cbrak}[1]{\ensuremath{\left\{#1\right\}}}
	\providecommand{\lcbrak}[1]{\ensuremath{\left\{#1\right.}}
	\providecommand{\rcbrak}[1]{\ensuremath{\left.#1\right\}}}
	\theoremstyle{remark}
	\newtheorem{rem}{Remark}
	\newcommand{\sgn}{\mathop{\mathrm{sgn}}}
	\providecommand{\abs}[1]{\(\left\vert#1\right\vert\)}
	\providecommand{\res}[1]{\Res\displaylimits_{#1}} 
	\providecommand{\norm}[1]{\(\left\lVert#1\right\rVert\)}
	%\providecommand{\norm}[1]{\lVert#1\rVert}
	\providecommand{\mtx}[1]{\mathbf{#1}}
	\providecommand{\mean}[1]{E\(\left[ #1 \right]\)}
	\providecommand{\fourier}{\overset{\mathcal{F}}{ \rightleftharpoons}}
	%\providecommand{\hilbert}{\overset{\mathcal{H}}{ \rightleftharpoons}}
	\providecommand{\system}{\overset{\mathcal{H}}{ \longleftrightarrow}}
	%\newcommand{\solution}[2]{\textbf{Solution:}{#1}}
	\newcommand{\solution}{\noindent \textbf{Solution: }}
	\newcommand{\cosec}{\,\text{cosec}\,}
	\providecommand{\dec}[2]{\ensuremath{\overset{#1}{\underset{#2}{\gtrless}}}}
	\newcommand{\myvec}[1]{\ensuremath{\begin{psmallmatrix}#1\end{psmallmatrix}}}
	\newcommand{\mydet}[1]{\ensuremath{\begin{vmatrix}#1\end{vmatrix}}}
	%\numberwithin{equation}{section}
	\numberwithin{equation}{subsection}
	%\numberwithin{problem}{section}
	%\numberwithin{definition}{section}
	\makeatletter
	\@addtoreset{figure}{problem}
	\makeatother
	\let\StandardTheFigure\thefigure
	\let\vec\mathbf
	%\renewcommand{\thefigure}{\theproblem.\arabic{figure}}
	\renewcommand{\thefigure}{\theproblem}
	%\setlist[enumerate,1]{before=\renewcommand\theequation{\theenumi.\arabic{equation}}
	%\counterwithin{equation}{enumi}
	%\renewcommand{\theequation}{\arabic{subsection}.\arabic{equation}}
	\def\putbox#1#2#3{\makebox[0in][l]{\makebox[#1][l]{}\raisebox{\baselineskip}[0in][0in]{\raisebox{#2}[0in][0in]{#3}}}}
	\def\rightbox#1{\makebox[0in][r]{#1}}
	\def\centbox#1{\makebox[0in]{#1}}
	\def\topbox#1{\raisebox{-\baselineskip}[0in][0in]{#1}}
	\def\midbox#1{\raisebox{-0.5\baselineskip}[0in][0in]{#1}}
	\vspace{3cm}
	\title{Assignment 8}
	\author{Addagalla Satyanarayana}
	\maketitle
	\newpage
	%\tableofcontents
	\bigskip
	\renewcommand{\thefigure}{\theenumi}
	\renewcommand{\thetable}{\theenumi}
\begin{abstract}
This document shows the diagonalization of matrices
\end{abstract}

%
\begin{lstlisting}
https://github.com/AddagallaSatyanarayana/AI5106/tree/master/Assignment8/Assignment8.tex
\end{lstlisting}
%
\section{Problem}
	Which of the following matrices are not diagonalizable over R
\begin{align}
\vec{M_1}&=\myvec{2&0&1\\0&3&0\\0 & 0&2}\\
\vec{M_2}&=\myvec{1&1\\1&1}\\
\vec{M_3}&=\myvec{2&1&0\\0&3&0\\0 & 0&3}\\
\vec{M_4}&=\myvec{1&-1\\2&4}
\end{align}

\section{Explanation}
A matrix $\vec{M}$ can be diagonalized ,if for every eigenvalue of $\vec{M}$, the geometric multiplicity equals the algebraic multiplicity, then A is said to be diagonalizable.
Eigen values can be calculated using
\begin{align}
	\mydet{\vec{M}-\lambda\vec{I}}=0\label{eq:2}
\end{align}
\subsection{Solving matrix1 }
\begin{align}
	\myvec{2&0&1\\0&3&0\\0 & 0&2}
\end{align}
The eigen values for the matrix $\vec{M}$ can be found using \eqref{eq:2}
\begin{align}
 \mydet{2- \lambda & 0 & 1\\ 0 & 3 -\lambda&0\\0& 0& 2-\lambda}=0
\end{align}
Solving this we get the eigen values of $\vec{M}$ as, 
\begin{align}
	\lambda_1 = 2\\
	\lambda_2=  3
\end{align}
The algebraic multiplicity of eigen value $ \lambda_1 = 2$ is equal to 2.

The eigen vector corresponding to eigen value $ \lambda_1 = 2$ 
\begin{align}
	\vec{v}=\myvec{x_1\\ x_2\\x_3}\\
	{(\vec{M}-\lambda_1\vec{I})\vec{v}}=0\\
	\myvec{2- \lambda_1&0&1\\0&3-\lambda_2&0\\0&0&2- \lambda_1}\myvec{x_1\\ x_2\\x_3}=0\\
	\myvec{0&0&1\\0&1&0\\0&0&0}\myvec{x_1\\ x_2\\x_3}=0
\end{align}
solve it by Gaussian Elimination
\begin{align}
	\myvec{0 & 0 & 1 \\ 0 & 1 & 0\\0&0&0}
	 \xleftrightarrow{R_2<->R_1}\myvec{0 & 1 & 0 \\ 0 & 0 & 1 \\0&0&0}\\
	 x_2=0\\
	 x_3=0\\
	 \vec{v}=\myvec{x_1\\ 0\\0}\\
	 \vec{v}=x_1\myvec{1\\ 0\\0}
\end{align}
where the scalar $x_1$ can be arbitrarily chosen. Therefore, the eigenspace of $\lambda _1$ is generated by a single vector
\begin{align}
	\vec{v}=\myvec{1\\ 0\\0}
\end{align}
Thus the geometric multiplicity of $\lambda _1=2$ is 1,but algebraic multiplicity of eigen value $ \lambda_1 = 2$ is equal to 2.
Since the geometric multiplicity and algebraic multiplicity are not same matrix $\vec{M}$ cannot be diagonalized.
\end{document}